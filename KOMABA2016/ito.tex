%まず最初に使ったプリアンブルをここに書いてください.
%ただしコンパイルの都合上コメントアウトしてください.
%実際に確認する際は,各自の環境でmain.texにこのプリアンブルを追加してください.

%\usepackage{mathrsfs}
%\usepackage[all]{xy}
%\newcommand{\proofend}{\begin{flushright} $\blacksquare$ \end{flushright}}
%\renewcommand{\labelenumi}{(\roman{enumi})}
%\newcommand{\nkgr}{・}
%\theoremstyle{definition}
%\newtheorem{theorem}{定理}
%\renewcommand{\thetheorem}{}
%\newtheorem{defi}{定義}
%\newtheorem{thm}[defi]{定理}
%\newtheorem{lem}[defi]{補題}
%\newtheorem{cor}[defi]{系}
%\newtheorem{prop}[defi]{命題}
%\newtheorem{ex}[defi]{例}


\Chapter{ネイピア数eについて(伊藤)}
\Section{はじめに}
\begin{flushright}
\textit {〜「私のことは嫌いでも、eのことは嫌いにならないでください!」〜}\\
国民的アイドルグループEXP271のとあるセンターの言葉
\end{flushright}

$e^{\pi i}sode$(えぴそーど)という名前の冊子を先輩たちから私達が受け継いでから,はや1年半が経ちましたが,未だに「$e^{\pi i}sode$?これはなんて読むの?」
と聞かれることも多いです.上記のような発言から察するに,$e^{\pi i}$にこめられた意味も理解されていないでしょう.
そこで改めて,{\bf e}という数字について色々とお話をしたいと思います.\par
$e$という数字は{\bf ネイピア数}と呼ばれ,日本語では自然対数の底とも呼ばれたりします.
数なのでちゃんと値があって,$e = 2.718281828459045235360287471352\cdots$という永遠につづく小数として表されます.
ネイピア数という名前は,この数を初めて発表したジョン・ネイピア(John Napier)という数学者にちなんでつけられました.
ネイピアがこの数を考案した1618年というのは,日本で言うと1615年は大阪夏の陣で徳川家康
\footnote{すごくどうでも良い話ですが,徳川家康を知らないと日本では非常識人呼ばわりされますが,ネイピア数eを知らない人に対して非常識だ!という石を投げる人はそこまでいないでしょう.残念です.} と豊臣秀頼が合戦を繰り広げていた年ですし,
ヨーロッパではまだ魔女狩りが勃発していたような時代です.
万有引力の法則を発見したアイザック・ニュートンはまだ生まれてもいませんし,ビブン,セキブンという概念・言葉すら生まれていません.
何が言いたいかと言いますと,「$e$は数学だけの世界に出てくる小難しい数ではなく,誰でも理解できるような身近なもの」ということです.\par
ということで,出来るだけ皆さんに分かりやすい様に$e$についてお話をしたいと思います.もし本当は身近なはずの$e$がこの冊子を読んで,
また遠くに感じられてしまったらそれは,$e$が悪いのではなく,私が悪いのです.

\Section{身近なe}
\begin{flushright}
\textit {〜「借りた金は忘れるな。貸した金は忘れろ。」〜}\\
田中角栄
\end{flushright}

$e$は身近なところから発生します.それは{\bf 金利}というところです.金利という概念は現代人にとってある意味遠い世界にあるものかもしれないので
一度説明しておきます.例えば,あなたが銀行にいくらかのお金を預けているとします.
\footnote{ここで「預けている」という言葉を使うと,まるで厳重な鍵付きの金庫やロッカーにお金を置いてきたように聞こえますが,そのようなことはありません.銀行も企業とは言え,その銀行の中にいるのは人間ですので,あなたのお友達の山田さんの家のタンスに「お金〜〜万円預けるね.なくしたら承知しないよ.」と言ってお金を置いてきたのと,解釈によっては変わりはないのです.}そうすると,銀行は「私達の所にお金を預けてくれてありがとう」と言って年に何度か少しだけあなたの口座にお金を振り込んでおいてくれます.これを{\bf 利子}と言います.\footnote{今の日本の銀行の通常預金の金利は0.001%ぐらいです.つまり,100万円預けて置くと年に10円ほどもらえる事になります.}ここで{\bf 金利(利率)}という概念が発生します.金利とは,「あなたが銀行に預けているお金」と「利子」の比率を金利と言います.数式で書くと
\[
\mbox{預金額}\times\mbox{金利} = \mbox{利子}
\]
となります.\footnote{ここまでの内容は,できれば読まなくても知っているぐらいのレベルであって欲しいです.}\par
ここからは,簡単のために全て年利率を{\bf 1%}の場合にのみ限定して話をすすめましょう.\\
もし,この世に年率$1$%を謳う銀行がいくつかあったとしましょう.これらは,皆同じに見えますが,実は違う可能性があります.
それは {\bf 年に何回利子が払われるかが分かっていない} ということです.例えば,
\begin{center}
100万円につき,年に一度だけ,1万円の利息がもらえる.
\end{center}
という銀行と,
\begin{center}
100万円につき,年に2度,5千円づつ利息がもらえる
\end{center}
という銀行は1年というスパンで見れば,おなじ年利率$1$%の銀行です.しかし,ここで違ってくるのが{\bf 複利}という考え方です.\par
(ここに複利の図を入れたい)
複利とは,今までもらった利息を預金額に繰り入れて,利息を払ってくれる方式です.例えば,100万円を(利息年1回払いの)年利息1%の銀行に$4$年間預けておくと
\begin{eqnarray*}
100\mbox{万円} \times 1\mbox{%}  + 100\mbox{万円} &=& 101\mbox{万円}\\
101\mbox{万円} \times 1\mbox{%}  + 101\mbox{万円} &=& 102.01\mbox{万円}\\
102.01\mbox{万円} \times 1\mbox{%}  + 102.01\mbox{万円} &=& 103.0301\mbox{万円}\\
103.0301\mbox{万円} \times 1\mbox{%}  + 103.0301\mbox{万円} &=& 104.0604\mbox{万円}
\end{eqnarray*}
というように雪だるま式お金が増えていきます.最初に考えた1年間の利払回数が違う場合についても同様のことがいえます.
\begin{center}
100万円につき,年に2度,5千円づつ利息がもらえる
\end{center}
は,
\begin{center}
$1$%$\div 2 = 0.5$%づつ半年に1回お金が増える
\end{center}
ということになります.そして,いくらになるかとか言いますと,

\begin{eqnarray*}
100\mbox{万円} \times 0.5\mbox{%}  + 100\mbox{万円} &=& 100.5\mbox{万円(=半年後の預金残高)}\\
100.5\mbox{万円} \times 0.5\mbox{%}  + 100.5\mbox{万円} &=& 101.0025\mbox{万円(=1年後の預金残高)}
\end{eqnarray*}
という風に$101$万円よりも,わずか$0.025$万円だけ増えました.つまり違う利息額となったわけです.\\
では,更に細かくわけて,$1$ヶ月に$1$回.年$12$回の利息が受け取れる銀行があったとしましょう.この場合は
\begin{center}
$1/12 \fallingdotseq 0.083$%づつ1ヶ月に1回お金が増える
\end{center}
ここで,今までのように$12$回計算をしても良いのですが,少し落ち着いて見てみると,\\
例えば$100$万円が$1$%増えるとその後どうなるかと言うのは,
\[
100\mbox{万円} \times 1\mbox{%}  + 100\mbox{万円} = 100\mbox{万円} \times (1 + 0.01) = 101\mbox{万円}\\
\]
という式で計算できるということがわかります.また,これを$2$回払いの式に応用すると
\[
100\mbox{万円} \times (1 + 0.005) \times (1 + 0.005) =  101.0025\mbox{万円}
\]
という風になります.同様に,$12$回払いの場合も
\[
100\mbox{万円} \times (1 + \frac{0.01}{12}) \times  \cdots \times (1 +  \frac{0.01}{12}) = 100\mbox{万円} \times (1 + \frac{0.01}{12})^{12} = 101.004596089
\]
という結果が得られます.\footnote{電卓で計算する場合は $1.00083333 \times\  = \ = \ = \ = \cdots $と$=$ボタンを連打すると計算できます}
また数字に着目すると,利息$12$回払いのときのほうがやはり僅かにお金は多くなっています.\\
では,もっとお金を増やしたい!!ということで,$1$日に$1$回利息が振り込まれるような銀行を考えてみるとどうでしょうか,
\[
100\mbox{万円}\times (1 + \frac{0.01}{365})^{365} = 101.005003 \mbox{万円}
\]
となってやはり,いままでよりも僅かに増えています.では,この{\bf 利払い回数をどんどん増やしていくと億万長者になれるのでは!!!}
\footnote{少し関連した話として,{\bf 72の法則}と言うものが有ります.これは,6%複利でお金を運用すると約12年で2倍になる,8%複利でお金を運用すると約9年で2倍になるというように
複利の%数と二倍になるまでの年数の積がおおよそ72になっているという法則です}
と思ってしまうわけです.例えば,年に$10000$回利払いがされるような銀行があったとしたら
\[
100\mbox{万円}\times (1 + \frac{0.01}{10000})^{10000} = 101.005016 \mbox{万円}
\]
となりますが,よく見てみると,そこまで増えていません.どうやら上限があるようです.そしてこの上限が$e$なのです.
ここで
\[
e^{0.01} = (2.718281828459045235360287471352 \cdots )^{0.01} = 1.01005016708
\]
という数値と見比べて見ましょう\footnote{流石にこれを普通の電卓で計算すわけにはいきませんので,関数電卓で計算するかGoogleで「e\^{}0.01」と検索してみてください}.
そうすると,実は$100$万円$\times e^{0.01} =  101.005016708$万円は今まで出てきた数字に非常に近い数字になっています.\par
つまり,経験的に,
\begin{center}
年利率$1$%の銀行の利払い回数をどんどん大きくしていくと,$1$年トータルでみたときには$e^{0.01}$という利率に近づく.
\end{center}
ということがわかります.これを高校数学の言葉で,{\bf $e^{0.01}$に収束する}と言います.こうして,$e$という数字が簡単な金利計算から出て来るということがわかりました.
この利率は,その瞬間瞬間に利息が発生し,それが預金に繰り入れられているという意味で{\bf 連続複利}と言われます.
(ここにeに収束していく図を入れたい)
\Section{まとめ〜eは本当に身近なのか〜}
\begin{flushright}
〜「A bird in the hand is worth two in the bush.」 〜\\
(手に持っている1羽の鳥は、まだ手にしていない茂みの中の2羽の鳥と同じ価値があるということわざ)。
\end{flushright}

利払い回数をどんどん大きくしていくと,やがては$e$に近づくということがわかりましたが,実際にそんなに何度も利払いが行われることなんてあるのだろうかと思う方もいらっしゃると思います.
しかし,1つ言えることは,$e^{利率}$は$1$日複利とほとんど差はないということです.そして,$e$という数字の計算上の便利さから,金融などの分野では$e$を使った利率計算が行われています.
そのことについて触れて,この記事を終えたいと思います.\par
{\bf どうして,金融機関では$e$を使う必要があるのか}というと,まず一つには$e$を使った利率計算が,数学的に相性が良いということがあります.\\
例えば,最初の年に$12$回の利払い$1$%,次の年に$6$回の利払い$2$%,そのまた次の年に$10$回の利払い$3$%の利率でお金を運用したときに,その利息の計算は
\[
(1+\frac{0.01}{12})^{12} \times (1+\frac{0.02}{6})^6 \times (1+\frac{0.03}{10})^{10}
\]
となり,いちいち全てを計算する必要があります.また,数学的に見てもこの式は複雑です.しかし,すべて連続複利であるとみなすと,
\[
e^{0.01} \times e^{0.02} \times e^{0.03} = e^{0.01+0.02+0.03}
\]
という風にシンプルな式にすることが出来ます.\footnote{全く関係がない話ですが,僕がお気に入りの問題として,「$e^\pi$に最も近い整数を電卓なしで計算せよ」という高校数学の問題が有ります.暇な方はやってみてください.もっと関係のない話として,$\pi^e,e^e,\pi^\pi$は全て有理数かすら分かっていませんが,$e^\pi$は超越数であることが分かっています.}またこの記事では述べませんが,この$e$の何々乗という数字はとても微分積分などの数学的な操作と相性が良いです.\par
最後に,そもそも全てのお金を銀行に預けているわけでもないのに,なぜ金融機関がこのようなことをしないといけないのかということについて考えてみましょう.これは,{\bf 今手にもっている100万円と 来年の100万円は当価値ではない}ということに由来しています.なぜならば,今の$100$万円を借りに銀行に預けたとしたら,$100$万円プラス利子がついているからです.
このようにして,金融機関では毎年のようにお金の出入りがありますから,それを一度全て現在の価値に換算して計算する必要があります.\par
「と言っても,今時超低金利だし関係ないんじゃないの」と思う方もいらっしゃるかもしれません.しかし,例えば生命保険や年金は非常に長期のお金の出入りがあります.
よってその積み重ねは大きく,少しでも金利が動いただけで,保険料や年金の掛金に大きな影響を影響をあたえることがあるのです.
\footnote{例えば,前年度のゼロ金利政策が日本銀行によって発表されたとき,多くの積立式の保険商品が販売をやめざるを得なくなったという事実からも分かると思われます}
故に,($e$を用いて)その会社のお金の出入りを管理していくことは非常に重要です.
\footnote{1つお詫びをしなければならないのは,eを用いた現在価値計算が行われるのはどちらかと言うと,数学色の濃い金融派生商品などの分野で,本文中で例として挙げた保険会社や年金などでは,一応「利力」という名前でもって認識はされていますが,少々数学との相性が悪くても,普通の複利計算でゴリ押ししてまうところがあります.}
またこれは,$e$が使われているほんの一例にすぎず,数学が絡む殆どの分野で$e$が使われているということを最後に注意しておきたいと思います.

\begin{thebibliography}{9}
\item Hull, J. C. (2014), Options, Futures, and Other Derivatives, 9th edition (Upper Saddle River, NJ: Prentice Hall).
\item Shoichiro Sakai「$C^★$-Algebras and $W^★$-Algebras」Springer
\item 梅垣壽春,大矢雅則,日合文雄「復刊 作用素代数入門」共立出版株式会社
\item 生西明夫,中神祥臣「作用素環入門I 函数解析とフォン・ノイマン環」岩波書店
\item 竹崎正道,「作用素環の構造」岩波書店
\item 綿村哲「非可換幾何学と場の理論」日本物理學會誌vol55, No10, 2000, 一般社団法人日本物理学会
\end{thebibliography}
