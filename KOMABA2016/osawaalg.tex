\Chapter{数学科で勉強すること(仮)(大澤)}
\Section{代数}
\Subsection{はじめに}
中学高校では代数というと、「数字だったものが文字に変わり、それらの計算を考える」という意味にとられることが多いでしょう。

$2 + 3 = 5$だったものが$2a + 3a = 5a$に置き換わる。人によってはとりとめのない変化だと感じる人もいるでしょう。
数学科の代数ではどのようなことが起きるかというと、「数字の一般化」だけでなく、「集合の一般化」「演算の一般化」が行われていきます。

数字が文字に変わったとはいえ、中学高校の頃はそいつらの正体は所詮``実数"の中のお話だったし、``足し算"や``掛け算"というのもよく知っているものでした。これらを一般化していくとどうなっちゃうの?といったお話を、これからしていこうと思います。

(ちなみに、``実数"ってなんだろう、``自然数"って「1とか2みたいなやつら」じゃ定義にならないよね、ということを考えるのも立派な数学ですよ!!)

%例えば,整数全体の集合と多項式の集合という2つの集合を見たときに,共通点として,足し算と掛け算はある.さらに割り算を考えることはできないが,商と余りを考えることはできる.という共通点がある.これは,代数学の言葉では,ユークリッド環とよばれるものであり,ユークリッド環について研究するということは,整数全体の集合の理解を深めることでもあり,多項式の集合全体の理解を深めるということでもある.
\Subsection{群とは}
群(ぐん)というのは、演算が入った集合に、最低限のルールを加えたものです。

最低限のルールさえあれば、別に集合の中身は実数じゃなくてもいいし、演算も足し算や掛け算じゃなくても構いません。変な話、「ペンとアッポーでアッポーペン、ペンとパイナッポーでパイナッポーペン」みたいな計算規則でもよいのです。

必要最低限の規則とは、「計算が集合の中で完結してること」、「結合法則が成り立っていること」、それから大事なのが「単位元と逆元の存在」です。例えば掛け算なら、どんな数でも1という数を掛け合わせれば結果は変わらないですよね。このように計算結果を変えないものを『単位元』といいます。また、0以外の数には逆数と呼ばれるものがあって、$3 \times 1/3 = 1$という風に計算結果を1にすることができますよね。これが『逆元』です。このとき、「0を除く実数は掛け算に関して群をなす」といいます。

掛け算じゃ面白くないので、別の例を考えます。「瀧くん、三葉ちゃん、奥寺先輩」の3人がいるとします。瀧君と三葉ちゃんが入れ替わることを$(瀧、三葉)$と書くことにします。さらに瀧君が入った三葉ちゃんと奥寺先輩が入れ替われば、瀧君に三葉ちゃんが、三葉ちゃんに奥寺先輩が、奥寺先輩に瀧君が入ることになります。この3人が順繰り入れ替わっている状況を、$(瀧、三葉)$と$(三葉、奥寺先輩)$の掛け算だと考えます。この入れ替わりの計算は、群をなします。「入れ替わらないこと」が単位元であり、また例えば$(瀧、三葉)$を2回やれば元に戻るので逆元もあります。この群を『対称群』といいます。

人数が増えても任意の入れ替わりは、このように二人の入れ替わりを繰り返せば実現できます。これが「あみだくじ」の原理です。あみだくじがどんな入れ替わりも再現してくれるのは、この対称群のおかげなのです。

ということで、いろんな集合、いろんな演算を無限に思いつくことができるのですが、たとえば「瀧くん、三葉ちゃん、奥寺先輩の入れ替わり」と「オバマ、クリントン、トランプ3人の入れ替わり」は、集合の元は違いますが実質的に同じですよね。もっと詳しく言うと、2つの群の間に演算を保つような1対1対応(写像)があるとき、2つの群は実質的に同じものであり、『群同型』であるといいます。群同型なものをまとめて扱うとき、たとえば元の数が100個の群は何種類あるか考える、という問題が思い浮かぶでしょう。

%群論はそれ単体でも奥が深い分野であり,その深遠さを表すものの1つに有限単純群の分類と言うものがある.有限単純群という群の中でも基本的な群は,「素数位数の巡回群」と「5次以上の交代群」と
%「リー型の16種類の群」と「26種類の散在型単純群」の4種類しか無いことが,2004年になってようやく分かった.これらは,それだけでも面白い結果であるが,更に散在型単純群の中で最も大きな群であるモンスター群(808017424794512875886459904961710757005754368000000000個の元を持っている)は物理学(共形場理論や弦理論)などとの関わりがあることが分かってきている.

\Subsection{環・体とは}
群で多少はっちゃけすぎたので、環・体は真面目にやります。

環・体には2つの演算が登場します。それぞれ、「足し算のような役割」と「掛け算のような役割」を果たします。これらにそれぞれ最低限の規則を与えて(例えば``足し算"の方は群の性質に加えて交換法則も必要だが、``掛け算"の方は逆元が必要ない)、分配法則が成り立つようにしたものを環(かん)といいます。さらに、``掛け算"にも交換法則が成り立ち、``割り算"もできるようにしたものを体(たい)といいます。

環になってくると、より``数字"に近いものを扱うようになります。整数や、$a + b\sqrt{2}$といった形の集合は環。有理数、実数、複素数のようなものは体になります。
%環のモデルとなっているものに,多項式全体の集合(多項式環)というものがある.なぜ多項式環を考えたかったか,ひいては環論を学ぶのかというと,幾何学者たちが「多項式によって定義される図形」
%($y=ax$で直線や,$x^2+y^2=1$で円など)を考えてきたということがある.つまり,環を考えることと多項式によって定まる図形を考えることは同じなのである.多項式環と幾何学のつながりを示した定理として,「ヒルベルトの零点定理」と呼ばれるものがある.これは,大雑把に言って「(アフィン)空間上の点全体」と「多項式環の極大イデアル」が一対一対応しているという定理であり,この定理によって,多項式環を考えることと幾何を考えることがおなじになった.これにより,多項式を考えることと多項式によって定義される幾何的対象を考えることの対応がつき,代数幾何学という分野が始まった.


体の``割り算"は完全に割り切ることになるのですが、環で``割り算"を扱うときは``商"``余り"を考えることになります。

高校数学で、整式の割り算を扱ったと思います。あれも``商"``余り"を考える整数のような扱い方をしましたが、じゃあ正式にも``素数"のようなものはあるのか。また、さっき挙げた$a + b\sqrt{2}$という集合は``素因数分解"できるのか。こういった概念を一般化するのが環の醍醐味です。

数字や多項式といったイメージしやすいものから、解析の世界に現れる作用素環というものまで、環の種類は様々です。群ほど自由度はありませんが、あらゆる場面に現れます。

%体論を学ぶ1つのモチベーションとして,「ガロア理論」と呼ばれるものがある.
%ガロア理論の1つ大きな結果として,「5次以上の方程式は一般には係数の四則演算と根号の組合せで解く事ができない」というものがある.
%ガロアの考え方は以下のようである.$x^2 = 2$という方程式は,有理数の中では解けないが,有理数に$\sqrt[]{2}$という数を足してあげれば解ける.
%つまり,考えている体を少しだけ大きくしてあげれば解ける.この体を少しだけ大きくする操作を拡大と言った.
%ガロアの注目すべきところは,「体の拡大」と「群」には密接な関係があることを見破ったことである.
%そして,「ある方程式$f$が係数の四則演算と根号の組合せで解く事ができる」という体の方程式の言葉を「$f$のガロア群が可解群である」
%という群論の言葉に置き換えた.これによって,「5次以上の方程式のガロア群が一般には可解群ではない」という群論の命題を証明すれば良いこととなり,
%ガロア理論は成功したのである.

\Subsection{体の上の線形空間}
といっても、群や環はかなり抽象的な数学で、数学が得意な学生であっても初めは戸惑う人の方が多いです。これらは数学科では、3年生の授業で扱います。

抽象数学の入り口として、理系の1,2年生は線形空間というものを扱います。高校数学でおなじみ『ベクトル』を扱うことになるのですが、$ベクトル = 矢印$というイメージからは離れてもらうことになります。ここでは、以下のような性質を満たせばベクトルということになります。

$K$ を体とするとき、 $V$ が $K$ 上のベクトル空間であるとは、任意の $K$ の元 $a$, $b$ と任意の $V$ の元 $u$, $v$, $w$ に対して
\begin{gather*}
(u + v) + w = u + (v + w) \\
u + v = v + u \\
ある0 \in V が存在して任意のvに対して v + 0 = v \\
任意のv \in V に対して (-v) \in Vが存在して v + (-v) = 0 \\
a(u + v) = au + av \\
(a + b)v = av + bv \\
a(bv) = (ab)v \\
1 \in K(Kの単位元)としたとき 1v = v
\end{gather*}
をみたすことである。



これは矢印だけでなく、例えば$ax^2 + bx + c$といった多項式も、$a$, $b$, $c$ を体の元とすればベクトルになります。

例えば空間のベクトルであれば、線形独立な3つの矢印を使ってすべての矢印を表現できますが、さっきのような多項式の話でも例えば $x^2$, $x$, $1$ に適当な数字をかけて足し合わせることですべての二次以下の多項式を表現できます。つまり、二次以下の多項式をすべて集めた集合は、ベクトル空間であるということができるのです。

引き続き多項式のベクトル空間を考えます。$x^2$ を微分すると $2x$, $x$ を微分すると1になりますが、これは二次以下の多項式の空間を一次以下の空間に移すことになります。このようにベクトル空間同士の写像を考えることになるのですが、これは行列を使って表されます(今の高校生は習わないようですが)。行列は数を2行2列、3行3列…などと並べたものなのですが、写像の性質を調べたいときに行列を使って話を進めることになります。行列の基本的な扱い方は1年生で学習します。

このようにベクトル空間を扱う学習を「線形代数」というのですが、数学科に内定が決まった2年生の後半まで続きます。そのころになってくると、ベクトル空間の間の写像を集めて、新しい空間として考える『双対空間』、2つのベクトル空間の``積"のようなものを考える『テンソル積』といったように、理解に時間がかかるような題材を扱うことになります。線形代数は大学数学の入り口だと言われることが多いですが、数学科でやる線形代数はかなり手ごわく、そのぶん極めれば抽象数学の魅力がたっぷり詰まっていると言えるでしょう。
