\Chapter{第一回 数学科意識調査!(大澤)}
数学科の人ってなに考えてるかわからない。数学のことしか考えてないの?そんなことを、よく言われてしまいます。
そこで、今回は数学科のB3・B4、計18人にアンケートをとり、数学科に通う学生の生態を探りました。読み物程度に眺めてみてください。

\Section{数学をやりたいと思ったのは?}
まずは、数学をやりたいと思ったきっかけをインタビュー。18人中、8人が「中学のころ」、5人が「高校のころ」と答え、全体の3/4を占めました。やはり、『数学』という科目の出会いが、運命を大きく変えたのでしょう。

\textbf{数学オリンピックとか部活(数学研究会)でのゼミとか(ペンネーム:simplicial object)}や、\textbf{ポアンカレ予想のドキュメンタリーを見て(普遍性)}など、学生時代からすすんだ数学の一端に触れていた人が、ここ数学科には多く集まっています。

中には、「小学校以前」から数学を志していた人も!\textbf{小学生の時にあたしンちに「因数分解」という単語が出てきて,調べてみたら面白かった.小6にやたらと塾の先生が「この問題は微積を使うと解ける」というので,インターネットで調べてみると面白かった.(後藤夢乃)}ということでまずは数学が好きになり、さらに\textbf{高校生のときに物理の先生と一緒に群論・環論の本をよんでいて面白かった.(後藤夢乃)}ということで現代数学にハマっていったと、熱く語ってくれました。

しかし、東大の特徴として「2年生の冬に学部・学科といった進路を決める」というものがあるので、当然大学生になってから数学に目覚めた人もいます。中には\textbf{昔から数学が好きだった事と、医学部での勉強に嫌気がさして(zatamura)}ということで、別の大学からこの東大理学部数学科に移ってきた人も!降年(希望の学科に行くため、1年留年すること)して、数学科を目指す人も少なくありません。そんな熱意を持った人があつまっているのです。

\Section{数学をやっててよかったことは?}
「サインコサインや、ビブンセキブンなんて、いつ使うの?」数学はしばしば「生きてて役に立たない」と思われがちです。そこで数学科の学生に、数学ができて(数学をやってて)よかったと思うことを聞いてみました。

\textbf{自分の頭の中をいかに人に伝えるかをとても考えるようになりました。(花屋のワルガキ)}というのが、とても印象的な回答でした。難しいことを難しく伝えるよりも、難しいものを噛み砕いて伝える方が何倍もエネルギーを使います。数学科でゼミなんかをやってると、こういったことはしょっちゅう鍛えられるのです。\textbf{論理的に物事を考えられるようになったこと(いぬい)}という回答もありましたが、このような力はたとえば就活でも活きてきます。

しかし大部分を占めるのは、\textbf{綺麗な証明や定理に出会って感動できる(GAP)}や、\textbf{単純に生涯ハマれるような趣味ができたこと。(ごま)}、\textbf{数学に関して真剣に議論できる友達が出来た}といった、数学の世界にどっぷりハマっているからこそ得られる喜びです。{話題が尽きたら適当な数学概念を説明することで会話が弾む(マスク)}、\textbf{等しいことと同型なこと違う概念だと気がつき、またそれに敏感になれた(こばけん)}というのは、既に学部生ながら研究者の風格があります。中には\textbf{特に無し(zetamura)}と断言してくれた人も。数学の世界には、数学を志す者にしかわからないよろこびが待っているのです。

最後に、\textbf{研究室に時間的に拘束されない}とか\textbf{多少面倒な説明書も楽々読めます}といったライフスタイルに関わるものもあったと報告しておきます。中でも、これは羨ましいと思う人も多いのでは…?
\textbf{中高生の頃は、数学を教えてほしいと女の子に囲まれた。その内の1人が今の彼女です$>\_<$(髭)}

\Section{数学以外のことを考えているときは?}
数学以外のことを考えているときは、どんなことを考えていますか?いくつかの選択肢を付けて聞いてみました。(複数選択可)

\textbf{眠い(12人)}というのが最多でした。大学生の性ですね。数学科の午前中の授業開始時刻は9時15分と、他の学科よりもやや遅いものの、それでも眠いものは眠いんです!仕方ない!

\textbf{お腹すいた(10人)}、\textbf{お酒が欲しい(7人)}、\textbf{彼女欲しい(7人)}が後に続いて、三大欲求に正直だなあと思いました。\textbf{バイト行かなきゃ(6人)}、\textbf{サークル行かなきゃ(4人)}という風に他に打ち込んでるものがある人も。バイトだと知り合いには塾・家庭教師だけでなく、喫茶店で働いてる人もいます。サークルは音楽系(特にオーケストラ)が多い印象です。

\textbf{アニメ観たい(5人)}は期待を裏切らないですね。学部生のフリースペース、学部生室には、アニメの原作になっている漫画(けいおん、キルミーベイベーや、聲の形まで)が揃っています。また。数学以外の勉強を楽しんでたり、\textbf{だいたいプログラミング(Ziphil)}という人もいます。さらに自由回答欄では筋肉をつけたい人が2人ほどいました。数学の問題も、筋肉ですべてを解決したいものです。

\Section{数学の魅力って?}
数学科の意識調査なのだから、せっかくなので数学の話もしましょう。専門にしたいことと、その魅力を語ってもらいました。
\textbf{視覚数理科学、応用なので役に立つ(淘汰)}という意見がありますが、これは筆者の私の専門も関わるところです。応用数理に近いような解析系は、数学科の中でも魅力を語りやすいと言われています。\textbf{色々な分野(解析学・確率論・応用数理など)の知識を活かすことができる.色々な分野(経済・物理等)に活かすことができる.}と答えてくれた人がいますが、数学科の有力な就職先の一つに経済分野があります。「数学ならだれにも負けない」という学生が、求められているのです。

解析を離れたところでは、\textbf{ホモトピー論、素朴なのに数学の言語って感じだ(simplicial object)}、\textbf{代数幾何 代数の人とも幾何の人とも話ができて楽しいです(たつろー)}と答えてくれました。中でも、数論(整数まわりの研究)をやろうとしている人は熱く語ってくれています。\textbf{数論 数の美しさを感じる(zetamura)}、\textbf{保形関数論:いい感じの関数が数論的情報を持ってくるのが神秘的(ばんほーてん)}といった風に。整数論は「数学の女王」と言われる分野ですが、熱意を持った研究者たちが女王に謁見すべく力を注いでいます。

数学の根幹を探ろうとする人も。\textbf{集合論; ZFCを越えた先に広がる世界を見ることができる(GAP)}詳しくは『公理的集合論』『ラッセルのパラドックス』で調べてもらうのがよいのですが、我々が数学を扱うときに基本となる『集合』。これが何かを表す公理系が『ZFC』(ツェルメロ=フレンケルの公理系に選択公理を加えたもの)だと言われています(現在これが使われているというだけで、もしかしたら覆されるかもしれない…)。ここで語るのはあまりに難しいので、ぜひ「ますらぼ」のブースにいる人に聞いてみてください。

\Section{さいごに・おまけ}
ここまで読んでいただきありがとうございました。数学科の人はこんなこと考えてるんだよ、こわくないんだよ、ということがお伝えできてればいいなと思います。
最後に、「無人島にひとつだけ何か持っていけるとしたら、どんな数学書を持っていきますか? 」という質問をしてみました。こんなふざけた質問に真面目に答えてくれる数学科のみんながとてもすきです。

幾何からは\textbf{categorical homotopy theory}, \textbf{位相幾何学}, \textbf{離散幾何学講義}。解析・応用系からは\textbf{complex analysis}, \textbf{ルディンのReal and complex analysis}, \textbf{Lambda Calculus with Types}。集合論だと\textbf{Kunenの集合論}, \textbf{Handbook of Set Theory (Volume 3)}。代数幾何が人気で、\textbf{ハーツホーンの代数幾何学}, \textbf{EGA(代数幾何原論)}がかなり票を集めていました。

もちろんちゃんと無人島に行くことを考えて、\textbf{食べられる数学書}, \textbf{EGA(燃料と食料として良さそう)}, \textbf{EGA…野生動物を殴る}, \textbf{解析系。解析系の式を砂浜に書き殴りたいですね。}というロマンある回答が見られました。最後の一人の回答は\textbf{持って行かない}…そりゃあそうだな。