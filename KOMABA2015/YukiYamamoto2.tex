\documentclass[b5paper,dvipdfmx]{jsarticle}
\usepackage{graphicx,color}
\usepackage{amsmath,amssymb,amsthm}
\usepackage{mathrsfs}
\DeclareMathOperator{\supp}{supp}
\DeclareMathOperator{\inter}{int}
\pagestyle{empty}
\addtolength{\topmargin}{-2zw}
\addtolength{\oddsidemargin}{-1zw}

%まず最初に使ったプリアンブルをここに書いてください.
%ただしコンパイルの都合上コメントアウトしてください.
%実際に確認する際は,各自の環境でmain.texにこのプリアンブルを追加してください.

%\usepackage{mathrsfs}
%\usepackage[all]{xy}
%\newcommand{\proofend}{\begin{flushright} $\blacksquare$ \end{flushright}}
%\renewcommand{\labelenumi}{(\roman{enumi})}
%\newcommand{\nkgr}{・}
%\theoremstyle{definition}
%\newtheorem{theorem}{定理}
%\renewcommand{\thetheorem}{}
%\newtheorem{ydefi}{定義}
%\newtheorem{ythm}[defi]{定理}
%\newtheorem{ylem}[defi]{補題}
%\newtheorem{ycor}[defi]{系}
%\newtheorem{yprop}[defi]{命題}
%\newtheorem{yex}[defi]{例}
\title{『Haar測度をつくろう』正誤表(誤植の鬼)}
%\author{\vspace{-5ex}}
%\date{\vspace{-8ex}}
\date{}
\begin{document}
\maketitle

この度は「数学科展示ますらぼ」に足をお運びいただき,そして冊子「$e^{\pi i}sode$」vol.3.5を手にとっていただき,誠にありがとうございます.

さて,この度私が$e^{\pi i}sode$に寄稿いたしました「Haar測度をつくろう」について,印刷後にあまりに多くの誤植が存在することが発覚いたしましたので,以下の通り正誤表を作成いたしました.お手数をおかけしますが,参照していただければ幸いです.

\begin{itemize}
\item 定義1の主張「$\mu(B)=\inf \{ \mu(O) \mid O \subset G : \mathrm{open} \}$」$\rightarrow$「$\mu(B)=\inf \{ \mu(O) \mid B \subset O \subset G : \mathrm{open} \}$」
\item 補題4証明中「${(F_i)^{c}}_{i \in I}$は$J$の開被覆」$\rightarrow$「$\{ (F_i)^{c} \}_{i \in I}$は$J$の開被覆」
\item 定理6の主張「$X \setminus V$上 $f \equiv 0$」$\rightarrow$「$X \setminus U$上 $f \equiv 0$」
\item 定理7の主張「$K$上 $\sum_{i=1}^{n}h_i \equiv 1$」$\rightarrow$「$K$上 $\sum_{i=1}^{n}f_i \equiv 1$」
\item 定義8の主張「$\supp f \in X$」$\rightarrow$「$\supp f \subset X$」
\item 定理9の主張「$\mu(B)=\inf \{ \mu(O) \mid O \subset G : \mathrm{open} \}$」$\rightarrow$「$\mu(B)=\inf \{ \mu(O) \mid B \subset O \subset G : \mathrm{open} \}$」
\item 命題11証明中「ゆえに$\| g\|  _{L^1(X)} < \| \chi_{K} \| _{L^1(X)} $ですが」$\rightarrow$「ゆえに$\| g\|  _{L^1(X)} < \| \chi_{K} \| _{L^1(X)} +\varepsilon $ですが」
\item 補題14証明中「定理6の$K$を$K$,$V$を$\inter(V)$として」$\rightarrow$「定理6の$K$を$K$,$U$を$\inter(V)$として」
\item 補題15証明中「$g(x)=f(x)+f(x^{-1})$なる$f' \in L_{+}(G)$」$\rightarrow$「$g(x)=f(x)+f(x^{-1})$なる$g \in L_{+}(G)$」
\item 命題16「$e$の近傍$V$であって($W$であって)」の後の$V_1 \rightarrow$ それぞれ「$V$」,「$W$」
\item 命題16証明中,第2段落「$x(a_k)^{-1}=x(y^{-1})y(a_k)^{-1} \in VW_{a_k} \in W_{a_k}W_{a_k}$」$\rightarrow$「$x(a_k)^{-1}=x(y^{-1})y(a_k)^{-1} \in VW_{a_k} \subset W_{a_k}W_{a_k}$」
\item 命題16証明中,第3段落「$xy^{-1} \in V_1$なる$x \in K, y \in G$」$\rightarrow$「$xy^{-1} \in V_2$なる$x \in K, y \in G$」
\item 命題18証明中,(5)「$\sup f \le \sup g ( \sum_{i}c_{i} )$」$\rightarrow$「$\sup f \le \sup g \cdot ( \sum_{i}c_{i} )$」
\item 命題18証明中,(6)「$\sum_{i} g( (s_{i} s^{-1}) ^{-1}x)$」$\rightarrow$「$\sum_{i} g( (s_{i} s_{0}^{-1}) ^{-1}x)$」
\item 命題20証明中「$F_{V_0} \neq \emptyset$」$\rightarrow$「$G_{V_0} \neq \emptyset$」
\item 命題20証明中「$g \in V_i$となるので」$\rightarrow$「$g \in G_{V_i}$となるので」
\item 命題23(2)の主張「$f$」,「$g$」$\rightarrow$それぞれ「$f_1$」,「$f_2$」
\item 命題23証明中,(2)第1段落「$(f_1 : g_m )+(f_1 : g_m)$」$\rightarrow$「$(f_1 : g_m )+(f_2 : g_m)$」 
\item 命題23証明中,(2)最終文「$I(f+g)=I(f)+I(g)$」$\rightarrow$「$I(f_1+f_2)=I(f_1)+I(f_2)$」
\item 命題25証明中,(Step2)で別行立て数式の最後「$\mu(O)$」$\rightarrow$「$\mu(sO)$」
\item 命題25証明中,(Step3)$\mathscr{M}$の定義「$\forall s \in G , sB \in \mathscr{B}(G)$」$\rightarrow$「$\forall s \in G , sA \in \mathscr{B}(G)$」
\item 命題27の主張「$f_0 \in L_{+}(G)$を固定すると,」$\rightarrow$「$f_0 \in L_{+}(G) \setminus \{ 0 \}$を固定すると,」
\item 命題27証明中,最初二つの別行立て数式「$Vx^{-1}$」$\rightarrow$「$V_{m}x^{-1}$」.また,「$d\mu (s)$」であるべきところが数か所「$\mu (s)$」となっている.
\item 命題27証明中,「$c_i=I(fh_i)/I(g_{m})$とおくと,」直後の別行立て数式中の「$s_i$」を削除.
\item 命題27証明中,(3)式直後「$\left( \sum_{i}c_{i}S_{i}g(x) : g_{m} \right) \le \sum_{i}c_{i}=I(f)/I(g)$」$\rightarrow$「$\left( \sum_{i}c_{i}S_{i}g(x) : g_{m} \right) \le \sum_{i}c_{i}=I(f)/I(g_{m})$」
\item 命題27証明中,(5)(6)式「$\left( f' : f \right) \left( f : g_{m} \right)$」,「$\left( f' : f_0 \right) \left( f_0 : g_{m} \right)$」$\rightarrow$「$\left( f' : g_{m} \right)$」
\end{itemize}

\end{document}
