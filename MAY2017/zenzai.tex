\documentclass[./main]{subfiles}

\begin{document}

\Chapter{三平方の定理はいつ生まれたか(ぜんざい)} 

直角三角形の三平方の定理(あるいはピュタゴラスの定理)は,数学を勉強する際に最も序盤に出会う最も美しい定理の一つです.この定理と証明は,紀元前3世紀ごろにユークリッドによって書かれた数学書『原論』第1巻命題47に出てきます.

実は,この三平方の定理が,いつ,誰によって見つけられたかということは,数学史の大きな未解決問題の一つです.本記事では,この定理にまつわる様々な説や謎のうちの一部を紹介したいと思います.

\Section{三平方の定理と古代エジプト}

ギリシア人は自分達の数学がエジプトから来たものだと思っていたようです.実際,巨大なピラミッド,年に1度の洪水のために高度に発達した測量術などがエジプト人の高度な学問水準に支えられていたことは疑いようもありません.しかしエジプトには定理の概念はありませんでしたから\footnote{エジプトの数学を伝える『リンド・パピルス』『モスクワ・パピルス』(どちらも紀元前18世紀から16世紀ごろ)では,賃金の分配や土地の面積の計算などの実際的な問題とその解答が書いてあります.},三平方の定理がそのままの形で知られていたと考えるには無理があります.

ただ,具体的な直角三角形についてなら彼らは三平方の定理を知っていた可能性があります.ドイツの数学史家M.カントルは「縄張り師」と呼ばれる役職が神殿の建立に関わっていることから,エジプト人は三辺の長さが3,4,5の三角形を知っていて,縄でこれを作ることで直角を作っていたのではないかという説を出しました.この説はかなり一人歩きをしていて,エジプト人は三辺の長さが3,4,5の三角形を知っているということを書いてある本やネットの記事はよく目にします.しかし,実際にエジプト人がこの直角三角形を知っていたという証拠は今のところ無いようです.

\Section{三平方の定理と古代バビロニア}

バビロニアとは,現在のイラク辺りをを指します.バビロニアでは,エジプトと同じくらい(あるいはそれ以上に)数学が栄えていました.

バビロニアの数学もエジプトと同じく,定理証明の形式ではありませんでしたから,三平方の定理がそのまま出てくることはありません.しかし,バビロニアの時代に既に三平方の定理が知られていたという説を主張する人は結構います\footnote{例えば[2]中村滋・室井和男(2014) 『数学史 数学5000年の歩み』 共立出版.}.実際,粘土板に出てくる問題や解法には,解答には直接関係ないが出てくる図形が三辺5,12,13の直角三角形だったり,計算の課程で三平方の定理が使われているように見えるところがあったりします.

とりわけその説を大きく広めたのは,バビロニア数学史研究の鏑矢的存在であるノイゲバウアーが1950年代に出した『プリンプトン 322』の解釈です.『プリンプトン 322』には文章がついておらず,ただ数が並べられて表になっています\footnote{実際がどんな感じか知りたい方は検索してみてください}.この数表がどうやって作られたのかを一目で見極めるのは難しいのですが,ノイゲバウアーはこれをうまく解釈して,これはピュタゴラス数\footnote{$a^2 + b^2 = c^2$を満たす3数($a,b,c$)のこと}を表しているのだという説を提示しました.およそ半世紀ほどの間,この説が正しく,バビロニア人は三平方の定理を知っていたのだということが通説になっていましたが,この数表をバビロニア人が2次方程式を解くときに使う公式から解釈する説もあり,現在はこちらが有力なようです.

実は,バビロニアに三平方の定理の起源を探ろうとするときの最大の困難がここにあります.バビロニアの数学専門家は,恒等式
\begin{equation}
	xy + \left( \frac{x-y}{2} \right)^2 = \left( \frac{x+y}{2} \right)^2
\end{equation}
をよく知っていて,2次方程式を解くときによく使っていました.ところがこれは$xy$が何かの数の2乗になったとき,三平方の式{$a^2 + b^2 = c^2$に他なりません.特に数だけが出てくる粘土板の場合だと,どっちのことを言っているのかはっきりしないのです\footnote{一方エジプトではこの公式は出て来ず,2次方程式も簡単なものしか解けていないようです.}.

\Section{三平方の定理とピュタゴラス}

私が中学生の時の数学の教科書には,ピュタゴラスが直角二等辺三角形が敷き詰められた模様を見たときに,この定理に気づいたというような説明が書いてあった記憶があります.他にも,ピュタゴラスがエジプト人やバビロニア人から数学を教えてもらっているときにこの定理に気づいたなど,この手の話はたくさんあります.ピュタゴラスが三平方の定理を見つけたとき,感動のあまりこの感謝の意を神々に示そうと犠牲を行ったそうなのですが,この犠牲は牛を1頭だったり,100だったり,小麦粉で作られた牛\footnote{\cite{Por} ピュタゴラスは不殺生主義者だったので生き物を犠牲に奉げることはことは無いという解釈がねじれてこのような記述が出て来たのだと思われますが,そもそもピュタゴラスが不殺生主義なのは人間の魂が生まれ変わる中で動物の中に宿っている可能性があるからであり,犠牲獣はこの輪廻から外れているため殺してもよいという解釈もあります.つまりよくわかりません.}だったりはっきりしません.

ピュタゴラスはサモス出身の人で,紀元前6世紀ごろに活躍しました.現代の私達が直接アクセスできるピュタゴラスの生涯や業績に関する言い伝えのほとんどはディオゲネス,ポリュピュリオス,イアンブリコス,プロクロス達の記述に依っています.彼らは全員紀元後の人達であり,特に後ろの3人は学派的にピュタゴラスびいきなため,彼らの話をそのまま鵜呑みにはできません\footnote{ちょうどこの記事のようなものです.}.

ピュタゴラスは,そのおよそ200年後を生きたプラトンやアリストテレス達にとっても既に謎の人物だったようです.彼らも,三平方の定理を示したのがピュタゴラスだと思っていたようで,実に2300年に渡ってこの定理はピュタゴラスのものだと思われていたのでした.この説への批判もあるにはあったのですが,それが決定的なものになったのは,1962年のブルケルトによる批判で,これ以降ピュタゴラスは天才数学者としてよりも天才宗教家として書かれることが多くなっています.三平方の定理を証明したのがピュタゴラスだという説も,論証的数学(つまり定義定理証明の形式)の成立が紀元前5世紀という説が有力になって来たため,あまり積極的には受け入れられてません.

個人的には,ピュタゴラスの数学への内容的な貢献が少ないとしても,ピュタゴラスは数学の進歩への偉大な功労者であるように思います.ピュタゴラスの教義を極めておおざっぱにいうと「数学的な知識を得ることで精神が昇華されるよ」というようなものです.例えば,現代では数学科なんかにいたりすると「どうして数学を勉強するのか?」というような疑問にぶつかったりぶつけられたりすることがあります.ピュタゴラスの信徒は少なくともこういう疑念に悩まされる必要は無かったわけです.三平方の定理を最初に厳密に証明したのがピュタゴラスで無いにしても,ピュタゴラスの死後分裂したピュタゴラス派の誰かである可能性が高いです.

\Section{三平方の定理と無理数}

現代に生きる私達にとって,三平方の定理から無理数の発見までは即時であるように思えます.直角二等辺三角形,正三角形を二等分したときに出て来る三角形などの重要な直角三角形が三辺の比に無理数を含むからです.三平方の定理の発見を早め早めにしようとする説の多くは,この考えとの両立の困難に苦しめられます.三平方の定理を知っていたのに,どうして無理数については何も知らないふうなのかと.

無理数について私達が遡ることができるのは,プラトンの対話篇『テアイテトス』までです.これが書かれたのは紀元前360年代,作中の年代は紀元前420年ごろとなっています.プラトンがもし嘘をついていないのであれば\footnote{プラトンの作品には,既に死んでいるはずの時代にソクラテスが登場するものもある(『メネクセノス』)ので,過信は禁物です},テアイテトスはこのころにいわゆる可術\footnote{ \cite{saito2} 2乗すると有理数になる無理数のこと.有理数面積の正方形の一辺の長さとして表せるからこの名前がつけられたのでしょうか.}な無理数を扱えるようになっていたということになります.この時のテアイテトスの語り方によると,彼がこの考えに気づくまでの無理数の理解はかなり未熟であったようです.そうだとすると,三平方の定理がピュタゴラスの時代に見つかっていたのに無理数の理解がそこまでしか進んでいないのは不思議だという考えが生まれます.この考えに従えば,三平方の定理の発見は早くとも紀元前5世紀ということになります.



\begin{thebibliography}{9}
\bibitem{sasaki} 佐々木力(2010) 『数学史』 岩波書店.
\bibitem{nakamura,muroi} 中村滋・室井和男(2014) 『数学史 数学5000年の歩み』 共立出版.
\bibitem{saito1} 斎藤憲(2008) 『ユークリッド『原論』とは何か』 岩波書店.
\bibitem{saito2} 斎藤憲(1997) 『ユークリッド『原論』の成立』 東京大学出版会.
\bibitem{Waerden} ヴァン・デル・ウァルデン(1984)『数学の黎明 オリエントからギリシアへ』 村田全・佐藤勝造訳,みすず書房.
\bibitem{Cajori} カジョリ(1997)『復刻版 初等数学史』小倉金之助訳,共立出版.
\bibitem{Por} ポルピュリオス『ピタゴラスの生涯』 水地宗明訳,晃洋書房.
\bibitem{Plato} プラトン『テアイテトス』田中美知太郎訳,岩波文庫.
\end{thebibliography}



\end{document}
