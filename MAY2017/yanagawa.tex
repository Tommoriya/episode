\documentclass[./main]{subfiles}

\usepackage{amsmath,amssymb,bm}
\usepackage[dvipdfmx]{graphicx}
\usepackage{float}

\newtheorem{thm}{命題}

\begin{document}

\Chapter{いかにして数学を説くか(柳川)}

\section{はじめに}

みなさんは数学を勉強されたことがあるでしょうか。この本を読んでくださっている方は数学になかなかの興味を持っていらっしゃる
かもしれません。しかし、この記事では数学が苦手な人,得意な人,好きではあるけどテストではそんなに点数が高くない人など
数学に触れたことのある方全員に読んでもらえるといいなと思っています。

この記事のテーマは,数学を学ぶ際に「分からない!」という気持ちになったときどう対処しようか、という点です。
後ほど具体例を出しますが,数学には学習者がもやもやする部分がたくさんあります。数学という学問自体はもやもやを
すべて取り除いて築き上げられているはずなのにこのような事態に陥るのはなぜでしょうか。一つの答えとして,次のようなものが
考えられるでしょう。「数学は,結局何がしたいのかがよく分からない。」筆者自身も頻繁に感じていました。
結局のところ,数学を理解する工夫を自ら試行錯誤するほかないのですが,その試行錯誤を少しでも読者の方に共有して
頂き,数学の学びの足を引っ張るものを取り除けたらなと思っています。

実はもう一つテーマがあり,それは「数学に関する質問をされたとき望まれる対応」です。筆者には文学部の弟がいますが,
その弟が大学受験の際ちょくちょく質問をしてくれました。それは数学に限ったことではなく,英語や古文,化学も多かったのですが,
その時に感じたのは「質問者の視点と回答者の視点がそろわないと本当の解決にはならない」ということです。
大事なことは,質問者の視点を想像するということであって,数学的に正しいことを述べ続ければいいというものではありません。
この「視点の揃え方」が上手くいけば,数学を他者と共有するという喜びが生まれます。
「分からない」というもやもやをもった人と,数学について話し合い,「なるほど!」と本心からいってもらえれば
これほど嬉しいことはありません。ですから個人的には,数学が得意な人には「数学を人に教えるのが上手な人」になってほしいと
思っています。そうなれば,世界が少しだけ平和になる気がします。

\section{数学でつまずくところ}

ここでは,筆者の経験から,数学でもやもやしたことのある具体例をいくつか挙げてみようと思います。数学といいつつ
算数も入っていますが似たようなものですのでお気になさらないようお願いします。

\subsection{分数}

小学校で次のような計算を習うでしょう。
\[
12 \div 3 = 4
\]
この計算は日本語で解釈することができ,「12個の石を3個ずつに分けると4グループできる」もしくは
「12個の石を3人の子どもに均等に分けると4個ずつになる」となります。一方で
\[
2 \div \frac{3}{5•} = \frac{10}{3•}
\]
という計算を日本語で解釈しようとしても「2個の石を5分の3個ずつ(?)分けると3分の10グループになる(??)」
となってしまって意味が通りません。

もう一つ似た例を挙げましょう。中学校で
\[
x^3 = x \times x \times x 
\]
という表記法を習います。これが高校になると
\[
x^{\frac{2}{3•}} = (x\text{を3分の2回かけたもの(?)})
\]
という怪しげなものに進化して,数学が分からないという気持ちを生むきっかけになっている気がします。

なぜもやもやするのかというと,「個」や「回数」といった,物事を数えるのに使う単位と,分数との相性がよろしくないからなのです。
もっと言えば,日本語にすると変なことになる概念に対してもやもやするのです。

\subsection{数学でよく出てくる文字}

なぜ数学では$x$とか$y$とか$\theta$とかがよく出てくるのでしょう。この手のいわゆる「数学の文化,慣習」に関する
疑問も多いと思います。

\subsection{高校数学と大学数学のギャップ}

大学に入って数学が途端に分からなくなる瞬間というものがあります。
大学数学の有名な障害物としては$\varepsilon - \delta$論法が挙げられるでしょう。
「極限(limit)」を厳密に論じるために必要だと説明されると思いますが,習いたての頃は
そんなことをする必要性がどこにあるのかという疑問が浮かび続けていたと思います。

大学数学では,必要とされる論理の厳密性が高校数学よりも格段に上昇します。そこに戸惑いを覚えるのだと思います。

\subsection{日本語と思えない日本語}

中学校辺りから関数,方程式という用語が登場します。高校では指数,対数や実数,複素数など。さらに大学では
写像,全単射,基底,一次独立,$n$回連続微分可能などなど。漢字で書いてはあるけど何も頭に入ってこない
専門用語の洪水です。その言葉の意味が分からないまま授業に置いて行かれるなんてことも多いと思います。

(少し蛇足ですが,大学3,4年生の数学を勉強しているとmodやhomoから始まる用語が多く登場します。module(加群),modulo(~を
法として),moduli(モジュライ),homotopy(ホモトピー),homology(ホモロジー),homogeneous(同次の)などです。
もはや日本語訳が作られていないものも多いのですが,これらの言葉が使用されている意図を理解するのは
容易ではないと感じています。)


\subsection{数学の本は行ったり来たりが激しい}

数学書を読もうとするとき,「この言葉はどういう意味だったっけ?」と数ページ前に戻り確認する,という作業が
とても多く発生します。正直めんどくさくなってやる気がなくなったりした経験はありませんか。

\section{分からない時の気持ち}

では,数学でもやもやしたときの原因を突き詰めてみましょう。そして,そのもやもやにどう立ち向かえばいいのかを
考えてみたいと思います。

\subsection{概念の拡張に出会った瞬間}
先ほど分数の例を出しました。ここには,新しい数を作るという「拡張」を受け入れられるかどうかという問題が絡んでいます。
分数を習うまではほとんど自然数(0,1,2,3,...)の世界で考えていたのに,3を5で割るという自然数だけでは表せない概念を
考えようとしているわけです。そこで,
\[
3 \div 5
\]
の答え,つまり「5を掛けたら3になる数」を
\[
\frac{3}{5•}
\]
と表すことにした,即ち「定義した」のです。自然数の世界から分数の世界に拡がったことが分かるでしょうか。
同様に,「3回掛けると$x^2$になる数」を
\[
x^{\frac{2}{3•}}
\]
と表すことにしたのです。指数(右肩に乗った小さい数字のこと)のところに自然数のみならず分数も使えるように世界が拡がっています。初めて習う時もこのように教わるはずですが,「なぜかわかった気にならない」のはなぜなのでしょう。

それは,それまでの自分の理解の仕方では解釈できないものに出会い,新しい理解の仕方が求められているからです。
これが,「日本語の解釈では上手くいかないのでよく分からない」という気持ちの正体です。ではどう対処するかというと,
筆者にもよく分かりません。ただ一つ言えるのは,新しい概念に出会ったときにもやもやするのは至極当然であって,
あなたが悪いわけでは決してないということです。おそらく,「このような理解の仕方があるのだな」ととりあえず受け入れて
みて,しばらく時間をおいてみると,いつの間にか自然なもののように思えてくるのだと思います。(マイナスがついた数も
いつの間にか受け入れていたでしょう?)

\subsection{そもそもなぜこんなものを考えるのか?}

分数の例で「概念の拡張」の例を挙げましたが,もう一つの顕著な例が実数から複素数への拡張です。
\[ 2乗して-1になる数をiとする\]
という文面を見たとき,「そんな数はないと思っていたのに急に何を言い出したんだ?」と思いませんでしたか。
その後「$a,b$を実数とするとき$a+bi$を複素数と呼ぶ」と習って,複素数の足し算,掛け算を学びます。
この頃に「なぜこんなものを考え始めたんだ」と悩む人が多いと想像します。

このようなもやもやは,何かを定義したときによく発生します。そしてこのもやもやが晴れるのは結構時間が経ってからである
ことが往々にしてあります。複素数の例でいうと,
\[
2次方程式は,複素数の範囲では必ず2個の解を持つ
\]
という定理を聞いたり,
\[
e^{i \theta} = \cos \theta + i \sin \theta
\]
という式を見たりした後に,どうやら複素数というものは役に立つものであるらしい,となんとなく思い始めるかもしれません。
筆者の場合は,高校物理で学ぶ交流回路の電流,電圧の式が複素数を用いるととても簡単に書けることに驚いて
複素数の有用性を初めて実感した覚えがあります。

概念を拡張するときのみならず,新しい定義に出会うときは「なぜこんなものを考えるのだろう」というもやもやが付きまといます。
その時はぜひ,そのもやもやを大事に取っておいたまま勉強を進めてほしいと思います。学び始めの頃はイメージが湧きにくく
自分が何をやっているのか分からなくなることもあると思いますが,その先に待つ応用を楽しみにして頑張るのが良いのでは
ないでしょうか。

もう一つ役に立つ解決策は,「他の人に聞く」ことです。返ってくる答えの大半は理解出来ず,どうやらすごいものらしいなあ
くらいの感想になってしまうのですが,たまに雷に打たれたように腑に落ちることがあります。

\subsection{なぜその文字を使うのか?}

学問にはその学問特有の文化,慣習というものがあり,数学にも当てはまります。その具体例を,第2章で挙げたものよりも
詳しく見てみましょう。

中学校で(現在は小学校でも?)次のような1次方程式を習います。
\begin{gather*}
2x + 3 = x - 1, \\
x = -4.
\end{gather*}
その時未知数は大抵$x$です。なぜ,他の文字や記号ではなく
$x$なのでしょう。実はこれは数学史の話題でして,誰も確たる答えを持っていません。少し調べてみると,アラビア起源説や
デカルト起源説などたくさん出てきます。デカルトとは「我思う,故に我あり」で有名な人ですが,彼の著作に$x$が出てくるのが
現存する最古の$x$らしいです。というわけで,「$x$を使うのはそういう文化なんだな」と諦めて受け入れてもらうほかありません。
そして,$x$の他にも未知数が出てくるときは,$x$の次のアルファベットである$y$を使うわけです。さらに未知数があれば$z$を
使うことが多いです。でも$z$より後のアルファベットはないのだから,4つ目の未知数が出てきたらどうするのでしょう。いちばん
多いのは$w$です'(筆者はこれを見てずるいと思いました)。ただ,5つ以上になると$x_1,x_2,x_3,x_4,x_5$のように右下に添え字を
付けてしまいます。そういう文化なのです。

この「諦めて受け入れる」という解決策は(十分悩んだ後ならば)そこそこ有用です。
なぜその記号を使うのか,という疑問は次のような記号に対してもよく向けられるでしょう。
\[
\sin \quad \cos \quad \tan \quad \sum \quad \int
\]
このような記号の起源を調べてみて下さい。「なあんだ割と適当なんだな」と思えるでしょう。そのうえで文化として受け入れて
もらうのがいいと思います。

他には,
\[点P,\quad 関数y=f(x),\quad 半径r, \quad 曲線C \]
という記号の割り当ても数学ではよく見ます。これらは対応する英単語の頭文字が使われている例です(点はpoint,
関数はfunction,半径はradius,曲線はcurve)。物理学でも時刻$t$,速度$v$,加速度$a$,力$F$などたくさん出てきます
ので調べてみてください。

繰り返しになりますが,この「なぜこの記号を使うのか」という疑問は大切に持ち続けて下さい。そのうちその記号の生みの親の
気持ちが分かるかもしれません。一度,「等しい」という意味の記号である
\[ = \]
の気持ちを考えてみてはどうでしょうか。

\subsection{「証明」について}

記号や文字以外にも数学特有の文化はたくさんあります。いちばん顕著なのは「証明する」という文化でしょう。
他の学問は,程度の差はあれど「うまい説明方法,解釈を生み出す」という目的があります。例えば物理学では
物理現象をうまく「説明できる」法則を探りますし,心理学では人心の動きをうまく「説明できる」因果関係を見出そうとします。
その点,数学は「説明」では飽き足らず,「証明」してしまおうというのです。
「説明」と「証明」の違いを見てみましょう。次の主張(数学では''命題''と呼ばれます)とその証明を見て下さい。(証明の
途中で「$\therefore$」という記号が出てきますが,これは「従って(therefore)」という意味の記号です。)

\begin{thm}
$a,b$を0より大きい実数とする。$a^2 = b^2$が成り立っているならば,$a = b$が成り立つ。
\end{thm}

(証明)
\[a^2 = b^2 \]
が成り立っているとする。両辺に$-b^2$を加えると
\begin{gather*}
a^2 + (-b^2) = b^2 + (-b^2), \\
\therefore \quad a^2 - b^2 = 0
\end{gather*}
となる。左辺を因数分解すると
\[
(a-b)(a+b) = 0
\]
となる。2つの実数$(a-b)と(a+b)$をかけて0になっているから,少なくとも片方は0と等しいので
\[
a-b=0, \quad または\quad a+b=0
\]
が成り立つ。従って,少し変形すると
\[
a=b, \quad または \quad a=-b
\]
が成り立つことが分かる。今,$a,b$はともに0より大きい実数なので,$a=-b$とはなり得ない。従って
\[
a=b
\]
が成り立つ。(証明おわり)

一方,この命題がなぜ成り立つかを「説明」しようとすると,例えば次のようになるでしょう。
\[
面積が等しい2つの正方形の,それぞれの1辺の長さは等しい。
\]
2つを見比べてどう感じられましたか。この例だと「説明」の方がよっぽど分かりやすいと思われたかもしれません。
ですが,2つの\.{長}\.{方}\.{形}の面積が等しいからと言ってそれぞれの2辺の長さが等しいわけではないことを考えると,
「説明」の方は正しいと確信するに足る根拠を人の直感に委ねていることが感じられませんか。

このように,数学における証明は必ずしも直感的に理解しやすいものではない代わりに,人の勝手なイメージに左右されない
厳密な論理で構成されます。数学を勉強する際,この証明の内容が分からなくて辛くなってくることが大変多いと思います。
そんな時はいったん深呼吸をして,証明のステップ1つ1つを自分の手で確かめながら進んでみましょう。ここに時間がかかるのは
当然,というよりはむしろここに時間をかけるのが数学の勉強のようなものです。1回の試みで分からなくても大丈夫です。
大抵は3回ほどアタックしてみてからじわじわ分かってくるものなのです。ここで必要なのは諦めない気持ちなので,
どうかじっくり取り組んでほしいと思います。大学数学の関門である$\varepsilon-\delta$論法も,じっくり腰を落ち着けて
向かい合ってみれば,希望が見えてくるはずです。3回ほどアタックしても厳しいときは,人に頼るのもよい選択肢です。

\subsection{数学の用語について}

数学には,日常でも使う言葉が数学用語になっていたり,逆に日常では使わないような単語が使用されていたりします。
例えば,「連続」という言葉は「3試合連続ホームラン」などでも使いますが,「関数が連続である」,という風に数学でも使います。
一方で,「3と5は互いに素である」とか「2と14は12を法として合同である」などのような,日常では聞くことの無い日本語が
使われます。(個人的に,このような謎の数学用語の極め付けだと思うのは「環」です。)
やはり,使い慣れない言葉を用いて勉強を進めるのは骨が折れます。このような時はどうすればいいのでしょうか。

一つの解決策は,「言葉を使い慣れる」ことです。そのための秘訣は,見慣れない言葉に出会ったとき,もしくはよく
意味を覚えていない単語に出会ったときの対応にあります。数学書には,初めて使う用語に対してはちゃんと「定義」
がしてあるはずなので,それをノートに書き写します。そして,その単語が出たときには,いったん自力で定義が思い出せるか
確認してみて,よく分からなければ(本を戻るのではなく)ノートを参照するようにして見て下さい。すると,何回か繰り返して
いるうちに単語の意味がしみ込んでくるでしょう。

英単語を覚えるときも,毎回意味が思い出せるか確認していたと思います。同じように,数学用語を見たときには
毎回定義が思い出せるかをチェックしてみてください。同じ本を行ったり来たりするのは面倒なので,別のノートに書き出して
参照できるようにしておくのです。コツは,本に出てくる順番通りにきれいにノートにまとめようと思ったりせずに,疑問に思った
順にドカドカノートに書き込んでいくことです。最初の内は,ある単語の定義に別の数学用語が含まれていたりして
芋づる式に単語を調べる羽目になると思いますが,そこを乗り越えればあなたの数学力はぐんぐん上昇するでしょう。

\subsection{自分が何をやっているのか分からない}

「何を計算しているんだこれは?」という状態になることがよくあります。その時は,何が与えられていて,何を知りたいのかを
紙に書き出してみるのがいいと思います。「図形が与えられていて,○○の長さを知りたい」「式が与えられていて,□□の
最大値を求めたい」など,最初は大雑把でもよいので目的意識をはっきりさせてみるとよいでしょう。授業中であれば,
先生に「今,何から出発して何をしようとしているんですか」と直接聞くのがよい方法です
(この質問はできるだけ早い方がよい)。数学の授業は(特に大学では)伏線を開けっ広げにしてある
推理小説のようなものですから,「この伏線は将来どう使われるのか」という質問は内容理解にとても役立つはずです。

「そもそもなぜ数学を勉強するんだ?」という疑問が頭から離れないときはいったん机から離れましょう。この疑問に対する
筆者なりの答えはこの記事の最後に回してあります。

\subsection{何が分かっていないのかが分からない}

「なんとなく納得していないんだけど,何が分かっていないのかうまく説明できない」という状態になることはとても多いでしょう。
そんな時は他の人に「分かっていない気がするところがあるから少し話を聞いてほしい」と頼んで\.静\.か\.に話を聞いてもらうのが
吉です。その時大切なのは,とりあえず自分が考えているところまでしどろもどろでもいいので話し切ることです。その後に
質疑応答を行い,勘違いを正したり,自分の理解と不理解の境界線を探りましょう。この状況で聞き手側に求められる
姿勢については第4章に書いてみました。

話を聞いてもらう相手がいない場合は,教科書などをいったん全部閉じて,白紙のノートに考えていることを書き出してみること
をお勧めします。結局のところ,「何を分かっていないのか」を言葉として外に出すことができれば半分問題は解決したような
ものなので,そこを目指しましょう。

\section{質問をされた側の目線}

数学について分からなくなった時,人に聞くことも有用だと何回か書いてきましたが,では質問を受けた側は
どう対応するのが良いのでしょう。ここからは2つ目のテーマである,「数学を伝える側に分かっておいてほしいこと」
について書こうと思います。筆者が数学について質問した時,された時の経験が元になっていますので一般論
とは程遠いですが,さらっと読み流して頂ければ嬉しいです。

\subsection{質問者の話を最後まで聞く}

質問する側は,質問をしている瞬間にも自分の分かっていないところが何なのか探し続け,整理し続けています。
なので,できれば中断せずに最後まで聞きましょう。質問の途中で「それはだって○○じゃないか」のように応答してしまうと,
質問者は整理しかけた問題点を崩して再度練り直さなくてはならなくなります。

\subsection{疑問点を一緒に炙り出す}

疑問解決の第一歩は,質問者のもやもやを回答者がしっかり共有することです。質問者が分かっているところから
始めて,順にステップを踏み,質問者の理解の最前線に回答者も立つことが必要です。

\subsection{質問者の知識と回答に必要な知識を擦り合わせる}

質問に対しての回答に,質問者の知らない事項が必要である場合がよくあります。その場合の回答の仕方には
注意が必要です。なぜかというと,「それを知らないんじゃ無理だよ」のような回答だと,質問者に「やはり自分は
数学が出来ないんだ」と思わせかねません。なので,ここまでは大丈夫か,ここまでは知っているか,聞いたことがあるか,
という具合に質問者の持っている知識を確認して,質問者の知識と回答をどうにかして繋げる作業が必要なのです。

\subsection{質問者の反応をよく見る}

質問者は,自分の分かっていないところを整理するのと同時に,回答者の言葉を咀嚼し理解を試みるという
なかなか複雑な思考を巡らせています。そこへ,回答者から言葉の洪水を浴びせてしまうと,質問者を
理解しきれない文章で溺れさせてしまうことになります。なので,回答者は一回発言するごとに質問者の反応を
よく見るようにしましょう。質問者が5秒くらい黙ってしまっても,我慢して待っていて下さい。疑問解決のための
質疑応答の主役は質問者側であることを常に意識するのが大切です。質問者に数学をしてもらうことが
肝要です。

\subsection{いきなり答えを出すのではなく,ヒントによって導く}

例えば,ピタゴラスの定理を使えば解ける問題があったとしましょう。そして,質問者が,ピタゴラスの定理に
気付いていなかったとしましょう。このとき,回答者側から「ピタゴラスの定理」と口に出してしまうのはヒントではなく
答えです。重要なのは,質問者に「ピタゴラスの定理を使えば良さげだ」と気付かせることであり,将来似たような
問題に出会ったときにも質問者の中にピタゴラスの定理を使う動機が生まれるように導くことです。
なのでこの場合のヒントとしては,長さの分かっている辺,分かっていない辺がどこか確認させたり,直角三角形
に気付くよう誘導するのが良いでしょう。井戸を掘るのではなく,井戸の掘り方を教えるのです。

\subsection{質問をしてくれたことに対する感謝を伝える}

これは筆者が嬉しかったので入れておきました。

\section{さいごに}
この記事のタイトルは,「How to Solve It(G.Polya著)」の邦訳である「いかにして問題を解くか(柿内賢信訳)」のタイトルを
もじらせて頂きました。筆者の拙い文章よりよっぽど読みやすく,しかもまとまっていますので,
ぜひ読んで頂きたいです(原著は70年も前に書かれていることに驚いています)。
他にも,「数学ガールシリーズ(結城浩著)」「数の悪魔ー算数・数学が楽しくなる12夜(エンツェンスベルガー著,丘沢静也訳)」
に影響を受けています。数学ガールシリーズからは,数学の対話や数学が分からないときの心情について,数の悪魔からは
数学を楽しく伝えることの魅力を学びました。この記事に魅力があるとは正直思えませんが,これをきっかけに今挙げた
3つの本のどれかでも手に取って頂ければ望外の喜びです。

筆者は一浪していまして,大学入試に落ちた時の数学の点数は120点満点中6点でした。ですがこうして
数学科で元気に数学を勉強しています。人と比べて自分は数学ができないと思う瞬間があると思いますが(無いなら
それはそれで素晴らしいです),そこで数学を勉強することを見限ってほしくありません。
記事の説明の中で「高校数学」「大学数学」という言葉を使っていましたが,本来数学は
そのような区別ができるものではありません。いつ,どこから勉強をしてもよい広大な学問ですから,地道に,
そして貪欲に勉強してもらえればと思います。筆者もそうしたいです。これまで書いてきた
数学におけるもやもやは,嫌悪されるものではなくむしろ大事に扱われるべきものであって,そのもやもやをきっかけに好奇心に
火をつけて勉強してもらえるといいなと思います。

「なぜ数学を勉強する必要があるのか?」この問いに対する筆者の現在の答えは,「自分の考えを表現する方法を
身に付けるため」です。数学には,概念を表現する手法が山のように含まれています。そして,表現された概念に
数回,数十回の手続きを加える(計算する)ことで,結論を得ます。このようにして得られた結論は,誰が見ても正しいもの
です。つまり,自分の頭の中をそのまま人に伝える手段の一つを数学は与えてくれているのです。筆者はまだ大学生という
若輩者なわけですが,自分の考えを人に伝えることの難しさと重要性を(主に部活の仕事で)痛感しています。
言葉を尽くしても尽くしても思ったように伝わっていないことは多々あります。自分の考えと相手の受け取り方を鑑みて
最適な表現法を探すのは,数学において概念を表現するのに似ています。むしろ人の心情を想像する方が数学よりも
よっぽど難しいのですが,深くまで突き詰めて考えて表現する思考回路は,自分の考えを表現するのにきっと役立つでしょう。
そして,他の人が表現しようとしていることを読み取ろうとするときにも大きな威力を発揮するでしょう。

筆者も数学科で勉強させて頂いているくらいには数学の世界に浸っていますので,伝わりにくいところも多い文章だったと
想像しますが,ここまで読んで下さってありがとうございました。

\end{document}
